%выбор типа документа -- статья; в квадратных скобках
%выбирается размер шрифта
\documentclass[12pt]{article}
%выбор размера полей
\usepackage[margin=1in]{geometry}
%здесь указаны пакеты без опций в квадратных скобках
\usepackage{setspace,float,amsmath,amsxtra,hyperref,authblk,amssymb,graphicx,multirow,xcolor,lscape,rotating,url,fancyhdr,multirow,tabularx,booktabs, amsthm}
\usepackage[utf8]{inputenc}
%подгрузка русского алфавита
\usepackage[russian]{babel}
\pagestyle{fancy}
\lhead{Методология и Методы Исследований в Социальных Науках}
\rhead{Евгений Седашов, 2025}
\hypersetup {
colorlinks=true,
linkcolor=cyan,
urlcolor=blue,
filecolor=magenta
}
\linespread{1.3}
%эти команды даны опционально для демонстрации
\newtheorem{thm}{Theorem}[section]
\newtheorem{lem}[thm]{Lemma}
\newtheorem{exr}[thm]{Exercise}
\makeatletter
\renewcommand*\env@matrix[1][*\c@MaxMatrixCols c]{%
	\hskip -\arraycolsep
	\let\@ifnextchar\new@ifnextchar
	\array{#1}}
\makeatother
\begin{document}
\begin{center}
\huge \textbf{Домашнее Задание 4} \\
\normalsize Дедлайн -- 12 января.  
\end{center}

\textbf{Задание 1.}

Напишите введение к Вашей курсовой работе по предложенной ниже структуре,  объёмом 2-3 страницы A4,  12 размер шрифта.  

\textbf{Структура для научно-исследовательского формата:}

\begin{itemize}
\item Актуальность. 
\item Исследовательский вопрос. 
\item Гипотеза или гипотезы.
\item Концептуализация и операционализация основных переменных.
\item Исследовательский дизайн.  
\item Предполагаемые источники данных,  описание предполагаемого итогового массива,  единица анализа.
\item Методы тестирования гипотез. 
\end{itemize} 

Если в методах будете писать про регрессии,  рекомендуем посмотреть,  какой тип модели подойдёт для Вашей зависимой переменной.  Тип модели будет связан с типом её распределения.  К более простым сюжетам (t-test,  ANOVA) тоже не стоит подходить ``автоматически'',  потому что не для всех зависимых переменных простые методы будут правильно работать. 

\textbf{Структура для проектного формата:}

\begin{itemize}
\item Назначение продукта -- рассказ,  какую потребительскую потребность Ваш продукт решает (возможен и вариант,  при котором Ваш продукт создаёт собственную нишу рынка,  тогда потребность,  условно говоря,  будет тоже новой); потребителями могут быть пользователи (B2C) или организации (B2B).  
\item Описание решаемой проблемы.  
\item Функционал продукта. 
\item Технические особенности ключевых функций продукта (например: рекомендательная система -- с помощью какого алгоритма?)
\item Существующий задел (что уже удалось сделать в рамках создания продукта?).   
\item Конкурентное поле и анализ предполагаемого рынка.
\end{itemize} 

\textbf{Задание 2}

\textit{Вариант 1:}

Для исследовательского трэка: напишите литобзор на 2 страницы A4,  с цитированием как минимум 10 статей по схожей или пересекающейся тематике.

Для проектного трэка: опишите детальную техническую реализацию продукта в виде блок-схемы ИЛИ разработайте методологию исследования для тестирования реакции потенциальных пользователей на функционал продукта (первый этап custdev).  

\textit{Вариант 2:}

Запрограммируйте на платформе Otree следующий двухфакторный эксперимент: 

1) Воздействие является комбинацией двух атрибутов,  происхождение богатства (природные ресурсы/инвестиции/IT-стартап) и его размер (большой/средний/небольшой),  само воздействие -- текст,  описывающий человека с конкретным набором атрибутов.  Каждый респондент получает 9 текстов (т.к.  9 комбинаций),  порядок текстов для каждого респондента случайный.  

2) Зависимая переменная -- ``Какая ставка налога на наследство (от 0 до 100) будет справедлива для людей, подходящих под описание выше?'',  вопрос появляется после каждого текста. 

3) Какая идея лежит в основе данного двухфакторого эксперимента? Зачем нужен второй фактор (размер богатства)? 

\textbf{Вариант 3}

Найдите любую статью,  где используется метод ``разность разностей'' или ``разрывная регрессия''.  Опишите статью по следующему алгоритму:

1) Какой исследовательский вопрос (или вопросы) ставится в статье? 

2) Какие гипотезы подвергатся тестированию? Опишите ключевые гипотезы и их обоснование.  Оцените представленные теоретические аргументы критически -- убедительна ли логика авторов? 

3) Опишите исследовательский дизайн (стратегию каузальной идентификации) и логику его обоснования авторами.  Считаете ли Вы аргументы авторов в пользу выбранного исследовательского дизайна убедительными? Обоснуйте свой ответ. 

\end{document}
